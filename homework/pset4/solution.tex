\documentclass[12pt]{exam}

\usepackage[utf8]{inputenc}  % For UTF8 source encoding.
\usepackage{amsmath}  % For displaying math equations.
\usepackage{amsfonts} % For mathematical fonts (like \mathbb{E}!).
\usepackage{upgreek}  % For upright Greek letters, such as \upvarphi.
\usepackage{wasysym}  % For additional glyphs (like \smiley!).
% For document margins.
\usepackage[left=.8in, right=.8in, top=1in, bottom=1in]{geometry}
\usepackage{lastpage} % For a reference to the number of pages.

% TODO: Enter your name here :)
\newcommand*{\authorname}{[Your name goes here]}

\newcommand*{\psetnumber}{4}
\newcommand*{\psetdescription}{Amortized Efficiency}
\newcommand*{\duedate}{Tuesday, May 21st}
\newcommand*{\duetime}{2:30 pm}

% Fancy headers and footers
\headrule
\firstpageheader{CS166\\Spring 2019}{Problem Set \psetnumber\\\psetdescription}{Due: \duedate\\at \duetime}
\runningheader{CS166}{Problem Set \psetnumber}{\authorname}
\footer{}{\footnotesize{Page \thepage\ of \pageref{LastPage}}}{}

% Exam questions.
\newcommand{\Q}[1]{\question{\large{\textbf{#1}}}}
\qformat{}  % Remove formatting from exam questions.

% Useful macro commands.
\newcommand*{\ex}{\mathbb{E}}
\newcommand*{\bigtheta}[1]{\Theta\left( #1 \right)}
\newcommand*{\bigo}[1]{O \left( #1 \right)}
\newcommand*{\bigomega}[1]{\Omega \left( #1 \right)}
\newcommand*{\prob}[1]{\text{Pr} \left[ #1 \right]}
\newcommand*{\var}[1]{\text{Var} \left[ #1 \right]}

% Custom formatting for problem parts.
\renewcommand{\thepartno}{\roman{partno}}
\renewcommand{\partlabel}{\thepartno.}

% Framed answers.
\newcommand{\answerbox}[1]{
\begin{framed}
\hspace{\fill}
\vspace{#1}
\end{framed}}

% MZ
\usepackage{amsthm}
\usepackage{amssymb}
\let\oldemptyset\emptyset
\renewcommand{\emptyset}{\text{\O}}
\renewcommand\qedsymbol{$\blacksquare$}
\newenvironment{prf}{{\bfseries Proof.}}{\qedsymbol}
\newcommand{\bi}[1]{\textit{\textbf{#1}}}
\newcommand{\annotate}[1]{\textit{\textcolor{blue}{#1}}}
\usepackage{stmaryrd}
\usepackage{pgfplots}
\pgfplotsset{compat=1.15}
\makeatletter
\@namedef{ver@framed.sty}{9999/12/31}
\@namedef{opt@framed.sty}{}
\makeatother
\usepackage{minted}
\usepackage{mathtools}
\usepackage{alltt}

\printanswers

\setlength\answerlinelength{2in} \setlength\answerskip{0.3in}

\begin{document}
\title{CS166 Problem Set \psetnumber: \psetdescription}
\author{\authorname}
\date{}
\maketitle
\thispagestyle{headandfoot}

\begin{questions}
%%%%%%%%%%%%%%%%%%%%%%%%%%%%%%%%%%%
\Q{Problem One: Stacking the Deque (3 Points)}

A \textit{deque} (\bi{d}ouble-\bi{e}nded \bi{que}ue, pronounced ``deck") is a data structure that acts as a hybrid between a stack and a queue. It represents a sequence of elements and supports the following four operations:
\begin{itemize}
\item \textit{deque}.\texttt{add-to-front}($x$), which adds $x$ to the front of the sequence.
\item \textit{deque}.\texttt{add-to-back}($x$), which adds $x$ to the back of the sequence.
\item \textit{deque}.\texttt{remove-front}(), which removes and returns the front element of the sequence.
\item \textit{deque}.\texttt{remove-back}(), which removes and returns the last element of the sequence.
\end{itemize}
Typically, you would implement a deque as a doubly-linked list. In functional languages like Haskell,
Scheme, or ML, however, this implementation is not possible. In those languages, you would instead implement
a deque using three stacks.

Design a deque implemented on top of three stacks. Each operation should run in amortized time $O(1)$. For the purposes of this problem, do not use any auxiliary data structures except for these stacks.

\begin{solution}

\textbf{Overview: }

The general idea is to use one stack to track the front of the .
\end{solution}

%%%%%%%%%%%%%%%%%%%%%%%%%%%%%%%%%%%
\newpage

\Q{Problem Two: Very Dynamic Prefix Parity (4 Points)}

On Problem Set Three, you designed a data structure for the \bi{dynamic prefix parity problem}. If you'll recall, this data structure supports the following operations:
\begin{itemize}
    \item \texttt{initialize}($n$), which creates a new data structure representing an array of $n$ bits, all initially zero. This takes time $O(n)$.
    \item \textit{ds}.\texttt{flip}($i$), which flips the $i$th bit of the bitstring. This takes time $O(\log n / \log \log n)$.
    \item \textit{ds}.\texttt{prefix-parity}($i$), which returns the parity of the subarray consisting of the first $i$ bits of the array. (the \textit{parity} is 0 if there are an even number of one bits and 1 if there are an odd number of 1 bits). This has the same runtime as the \texttt{flip} operation, $O(log n / log log n)$.
\end{itemize}

Now, let's consider the \bi{very dynamic prefix parity problem}. In this problem, you don’t begin with a fixed array of bits, but instead start with an empty sequence. You then need to support these operations:
\begin{itemize}
    \item \textit{ds}.\texttt{append}($b$), which appends bit $b$ to the bitstring. 
    \item \textit{ds}.\texttt{flip}($i$), which flips the $i$th bit of the bitstring.
    \item \textit{ds}.\texttt{prefix-parity}($i$), which returns the parity of the subarray consisting of the first $i$ bits of the array.
\end{itemize}
Design a data structure for the very dynamic prefix parity problem such that all three operations run in \textit{amortized} time $O(\log n / \log \log n)$, where $n$ is the number of bits in the sequence at the time the operation is performed. 

\begin{solution}
Your solution goes here!
\end{solution}

%%%%%%%%%%%%%%%%%%%%%%%%%%%%%%%%%%%
\newpage

\Q{Problem Three: Palos Altos (3 Points)}

The order of a tree in a Fibonacci heap establishes a lower bound on the number of nodes in that tree (a tree of order $k$ must have at least $F_{k+2}$ nodes in it). Surprisingly, the order of a tree in a Fibonacci heap does \textit{not} provide an upper bound on how many nodes are in that tree.

Prove that for any natural numbers $k$ and $m$, there's a sequence of operations that can be performed on an empty Fibonacci heap that results in the heap containing a tree of order $k$ with at least $m$ nodes.

\begin{solution}
Your solution goes here!
\end{solution}

%%%%%%%%%%%%%%%%%%%%%%%%%%%%%%%%%%%
\newpage

\Q{Problem Four: Meldable Heaps with Addition (12 Points)}

Meldable priority queues support the following operations:
\begin{itemize}
\item \texttt{new-pq()}, which constructs a new, empty priority queue;
\item $pq$.\texttt{insert}($v$, $k$), which inserts element $v$ with key $k$;
\item $pq$.\texttt{find-min()}, which returns an element with the least key;
\item $pq$.\texttt{extract-min()}, which removes and returns an element with the least key; and
\item \texttt{meld}($pq_1, pq_2)$, which destructively modifies priority queues $pq_1$ and $pq_2$ and produces a single priority
queue containing all the elements and keys from $pq_1$ and $pq_2$.
\end{itemize}

Some graph algorithms, such as the Chu-Liu-Edmonds algorithm for finding the equivalent of a minimum spanning tree in directed graphs (an \textit{optimum branching}), also require the following operation:
\begin{itemize}
\item $pq$.\texttt{add-to-all}($\Delta k$), which adds $\Delta k$ to the keys of each element in the priority queue.
\end{itemize}
Using lazy binomial heaps as a starting point, design a data structure that supports all \texttt{new-pq, insert,
find-min, meld}, and \texttt{add-to-all} in amortized time $O(1)$ and \texttt{extract-min} in amortized time $O(\log n)$.

Some hints:
\begin{enumerate}
\item You may find it useful, as a warmup, to get all these operations to run in time $O(\log n)$ by starting
with an \emph{eager} binomial heap and making appropriate modifications. You may end up using some
of the techniques you develop in your overall structure.
\item Try to make all operations have worst-case runtime $O(1)$ except for \texttt{extract-min}. Your implementation
of \texttt{extract-min} will probably do a lot of work, but if you've set it up correctly the
amortized cost will only be $O(\log n)$. This means, in particular, that you will only propagate the
$\Delta k$'s through the data structure in \texttt{extract-min}.
\item If you only propagate $\Delta k$'s during an \texttt{extract-min} as we suggest, you'll run into some challenges
trying to \texttt{meld} two lazy binomial heaps with different $\Delta k$'s. To address this, we recommend that
you change how \texttt{meld} is done to be even lazier than the lazy approach we discussed in class. You
might find it useful to construct a separate data structure tracking the \texttt{meld}s that have been done
and then only actually combining together the heaps during an \texttt{extract-min}.
\item Depending on how you've set things up, to get the proper amortized time bound for \texttt{extract-min},
you may need to define a potential function both in terms of the structure of the lazy binomial
heaps and in terms of the auxiliary data structure hinted at by the previous point.
\end{enumerate}

As a courtesy to your TAs, please start off your writeup by giving a high-level overview of how your data
structure works before diving into the details.

\begin{solution}
Your solution goes here!
\end{solution}

%%%%%%%%%%%%%%%%%%%%%%%%%%%%%%%%%%%
\Q{Problem Five: Splay Trees in Practice (8 Points)}

This one is all coding! Make sure to submit your code on myth.

\end{questions}
\end{document}